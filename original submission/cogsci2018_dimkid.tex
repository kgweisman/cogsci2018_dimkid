% Template for Cogsci submission with R Markdown

% Stuff changed from original Markdown PLOS Template
\documentclass[10pt, letterpaper]{article}

\usepackage{cogsci}
\usepackage{pslatex}
\usepackage{float}
\usepackage{caption}

% amsmath package, useful for mathematical formulas
\usepackage{amsmath}

% amssymb package, useful for mathematical symbols
\usepackage{amssymb}

% hyperref package, useful for hyperlinks
\usepackage{hyperref}

% graphicx package, useful for including eps and pdf graphics
% include graphics with the command \includegraphics
\usepackage{graphicx}

% Sweave(-like)
\usepackage{fancyvrb}
\DefineVerbatimEnvironment{Sinput}{Verbatim}{fontshape=sl}
\DefineVerbatimEnvironment{Soutput}{Verbatim}{}
\DefineVerbatimEnvironment{Scode}{Verbatim}{fontshape=sl}
\newenvironment{Schunk}{}{}
\DefineVerbatimEnvironment{Code}{Verbatim}{}
\DefineVerbatimEnvironment{CodeInput}{Verbatim}{fontshape=sl}
\DefineVerbatimEnvironment{CodeOutput}{Verbatim}{}
\newenvironment{CodeChunk}{}{}

% cite package, to clean up citations in the main text. Do not remove.
\usepackage{cite}

\usepackage{color}

% Use doublespacing - comment out for single spacing
%\usepackage{setspace}
%\doublespacing


% % Text layout
% \topmargin 0.0cm
% \oddsidemargin 0.5cm
% \evensidemargin 0.5cm
% \textwidth 16cm
% \textheight 21cm

\title{Folk philosophy of mind: Changes in conceptual structure between 4-9y of
age}


\author{{\large \bf Kara Weisman} \\ \texttt{kweisman@stanford.edu} \\ Department of Psychology \\ Stanford University \And {\large \bf Carol S. Dweck} \\ \texttt{dweck@stanford.edu} \\ Department of Psychology \\ Stanford University \And {\large \bf Ellen M. Markman} \\ \texttt{markman@stanford.edu} \\ Department of Psychology \\ Stanford University}

\begin{document}

\maketitle

\begin{abstract}
We explored children's developing understanding of mental life using a
novel approach to track changes in conceptual structure from the bottom
up by analyzing patterns of mental capacity attributions. US children
ages 4-9y (\emph{n}=247) evaluated elephants, goats, mice, birds,
beetles, teddy bears, dolls, robots, and computers on a range of mental
capacities, allowing us to assess which attributions ``go together'' and
how these conceptual connections might develop over early and middle
childhood. Replicating previous studies with adults and older children,
an exploratory factor analysis of older children's (7-9y) responses
revealed a three-way distinction between physiological abilities (e.g.,
hunger, smell), social-emotional abilities (e.g., guilt, embarrassment),
and perceptual-cognitive abilities (e.g., choice, memory), corresponding
to traditional notions of BODY, HEART, and MIND. In contrast, younger
children (4-6y) integrated both physiological and social-emotional
capacities into a single category---what we might call
BODY-HEART---distinct from the perceptual-cognitive abilities of the
MIND.

\textbf{Keywords:}
mind perception; conceptual change; lay biology; lay psychology;
cognitive development.
\end{abstract}

\section{Introduction}\label{introduction}

From early in life, attributions of mental capacities govern our
interactions with other beings and inform our judgments about their
moral status. In order to understand, predict, and coordinate with
others, we make constant inferences about their thoughts, feelings, and
other aspects of mental life.

Developmental and cognitive psychologists have made great progress in
understanding how people make sense of other minds by postulating
distinct representations of such categories as ``perceptions,''
``beliefs,'' ``desires,'' and ``emotions.'' These categories have been
incredibly useful---but do they correspond to children's own developing
understanding of the structure of mental life?

After all, mental life is extremely complex. Consider a few examples of
the many dimensions that might organize this conceptual space: Some
mental capacities are closely related to specific bodily organs (e.g.,
vision, hunger), and others less obviously so (e.g., belief); some are
positively or negatively valenced (e.g., pain, happiness), and others
more neutral or variable (e.g., smell, thought); some involve taking in
information about the environment, while others involve storing,
updating, or using that information to bring about changes in the
external world; some are broadly similar across species, and others may
be unique to humans. How do people of different ages conceive of the
connections and distinctions among mental states, and how does this
conceptual structure shape their understanding of the various humans,
animals, and technologies in their world?

In recent studies, we have set out to derive this conceptual structure
empirically, using an approach---exploratory factor analysis---that
reveals the conceptual connections and distinctions underlying
participants' responses from the bottom up. Inspired by Gray et al.'s
(2007) work on the ``dimensions of mind perception,'' we first used this
bottom-up approach to analyze patterns of mental capacity attributions
among US adults. Across several studies, three suites of mental
capacities consistently emerged: (1) A suite of capacities related to
the BODY, including physiological sensations and self-initiated
behaviors; (2) a suite of capacities related to the HEART, including
social-emotional experiences and moral agency; and (3) a suite of
capacities related to the MIND, including perceptual-cognitive abilities
and goal pursuit (Weisman et al., 2017b). A further study with 7- to
9-year-old children using an adapted version of this experimental
paradigm suggested that this distinction between BODY, HEART, and MIND
might be in place by middle childhood (Weisman, Dweck, \& Markman,
2017a).

Here we extend this approach down to younger children to explore the
earlier development of this conceptual system. The preschool years are a
time of rapid conceptual change in the domain of lay psychology, as
evidenced by dramatic shifts in children's abilities to take others'
perspectives, represent false beliefs, reason about unconventional
opinions, and integrate representations of intentions and outcomes in
evaluating moral responsibility (for reviews, see Flavell, 1999;
Wellman, 2015). The period between 4-10y of age has also been the focus
of a rich tradition of work on lay biology extending back nearly a
century (e.g., Carey, 1985; Medin, Waxman, Woodring, \& Washinawatok,
2010; Piaget, 1929). In the current paper, we focus on children in two
age groups---4-6y and 7-9y---and aim to assess the similarities and
differences in their representations of the wide range of sensations,
perceptions, beliefs, thoughts, desires, and emotions that might be
considered part of a folk philosophy of mind.

\section{Study}\label{study}

We based our experimental paradigm on our previous work with children
ages 7-9y (Weisman et al., 2017a), in which children evaluated a target
character on a variety of mental capacities using a 3-point response
scale (\emph{no}, \emph{kinda}, \emph{yes}). Although a 3-point scale is
not optimal for factor analyses, it enabled children as young as 4y to
answer questions comfortably and complete many trials. Including a wide
age range while maintaining a within-subjects design was our top
priority for the planned factor analyses.

As in previous work with adults (Weisman et al., 2017b, Study 4), we
asked each participant to judge the mental capacities of one target
character out of a set of familiar entities. We included both animals
and artifacts in this set, with an eye toward exploring age-related
differences in the relationship between attributions of biological
animacy and mental life (see, e.g., Carey, 1985; Gelman \& Opfer, 2002).
For animals, we included both mammals of different sizes and
relationships to humans (elephant, goat, mouse) and non-mammals (bird,
beetle), to represent a range of creatures that might vary in their
perceived mental capacities. For artifacts, we included both
anthropomorphic toys (teddy bear, doll) and ``smart'' technologies
(robot, computer), which present different kinds of challenges in terms
of grappling with the relationship between animacy, pretense, and mental
life. Most critically for our bottom-up approach, we expected these
target characters to be perceived as having very different mental lives:
Robots, for example, are generally thought to have a different set of
mental capacities than, say, goats (Weisman et al., 2017b). This allowed
us to address the following question: When different characters are
thought to have different profiles of mental capacities, which
capacities ``go together''?

\subsection{Methods}\label{methods}

\subsubsection{Participants}\label{participants}

247 children participated in this study, which took place in the San
Francisco Bay Area. Our planned sample size was 120 older and 120
younger children, but we also retained a handful of extra participants
who completed the study on the final day of data collection for each
group. Older children (\emph{n}=123) ranged in age from 7.09-9.99y
(median: 8.57y), and participated at local museums; the median study
duration for this group was 2.70min. Younger children (\emph{n}=124)
ranged in age from 4.00-6.99y (median: 5.03), and participated either at
their preschool (68\%) or at a museum (32\%); the median study duration
for this group was 3.58min. An additional 7 children participated but
were excluded for being outside the target age range.

\subsubsection{Materials and procedure}\label{materials-and-procedure}

Participants were assigned to evaluate one of the following characters:
elephant, goat, mouse, bird, beetle, teddy bear, doll, robot, or
computer (\emph{n}=10-16 per character, per age group). Participants
were assigned to condition randomly, with two exceptions: The doll and
teddy bear conditions were run last for older children (but included in
the initial randomization scheme for younger children); and toward the
end of data collection for each age group children were assigned to
conditions that had the fewest participants. A vivid, high-resolution
photo of the target character in a naturalistic context (e.g., a
humanoid robot in an office) and a label (e.g., a robot) were displayed
on a computer screen for the duration of the study.

Instructions were identical to previous work with children (Weisman et
al., 2017a), focusing on the idea that we wanted to know what children
thought (e.g.) ``{[}robots{]} can do and can not do.'' Children rated
the target character on 20 mental capacities, presented in a random
order for each participant. On each trial, children responded \emph{no},
\emph{kinda}, or \emph{yes} to the question ``Do you think a {[}robot{]}
can\ldots{}?''

The 20 mental capacities were a subset of the 40 items used in previous
work with children (Weisman et al., 2017a)., including physiological
sensations related to biological needs, emotional experiences,
perceptual abilities, cognitive skills, capacities related to autonomy
or agency, and social abilities; see Figure 1. As in previous work, each
mental capacity was associated with a short, preset definition. Children
were encouraged at the beginning of the study to ask questions if they
did not know what a word meant, in which case they given these
definitions.

\subsubsection{Data preparation}\label{data-preparation}

We scored responses of \emph{no} as 0, \emph{kinda} as 0.5, and
\emph{yes} as 1. We planned to drop trials with response times that were
faster than a preset criterion of 250ms, but there were none. We
retained participants regardless of skipped trials (0 trials among older
children, 30 trials among younger children). Overall, none of older
children's trials and only 1.21\% of younger children's trials were
missing data.

\subsubsection{Planned analyses}\label{planned-analyses}

Following previous work (Weisman et al., 2017a, 2017b), we conducted
exploratory factor analyses (EFA) to reveal the latent structure
underlying participants' mental capacity attributions, collapsing across
characters and using Pearson correlations to find minimum residual
solutions. We first examined maximal (6-factor) solutions to determine
how many factors to extract, using the following preset retention
criteria (identical to Weisman et al., 2017a, 2017b): Each factor must
have an eigenvalue \textgreater{}1.0 and individually account for
\textgreater{}5\% of the total variance before rotation; and each must
be the ``dominant'' factor (have the highest factor loading) for
\(\geq\) 1 mental capacity after rotation. We used an oblique rotation
(oblimin) here because it allows us to examine correlations among
factors; note, however, that constraining factors to be orthogonal (via
varimax rotation) yielded very similar latent structures.

\subsection{Results and discussion}\label{results-and-discussion}

We first assess the conceptual replication of our previous work with 7-
to 9-year-old children by conducting EFA of older children's responses.
We then examine this conceptual system at an earlier point in
development via EFA of younger children's responses. Finally, we present
a post-hoc analysis of individual children's endorsements of three
categories of mental capacities: physiological, social-emotional, and
perceptual-cognitive. This provides a more intuitive picture of the EFA
results and sheds new light on how the co-occurrence of endorsements
across these three categories might vary with age.

\subsubsection{EFA: Older children
(7-9y)}\label{efa-older-children-7-9y}

EFA revealed 3 factors that met our retention criteria. (Alternative
approaches to factor retention---parallel analysis and minimizing
BIC---also yielded 3 factors.) See Figure 1 (columns 1-3) for the full
results of this analysis.

After rotation, the first factor corresponded primarily to physiological
sensations and other experiences related to biological needs and
physical survival. It was the dominant factor for such items as
\emph{get hungry, smell things, feel scared, feel pain}. The second
factor corresponded primarily to social-emotional experiences. It was
the dominant factor for such items as \emph{feel guilty, feel
embarrassed, feel proud, get hurt feelings}. The third factor
corresponded primarily to perceptual and cognitive abilities to detect,
store, and make use of information about the environment. It was the
dominant factor for such items as \emph{figure out how to do things,
make choices, remember things, sense temperatures}.

This provides a conceptual replication of our previous work with this
age group, suggesting that this conceptual structure emerges not only
when children are asked to reason about controversial ``edge cases''
(beetles and robots) and factors are constrained to be orthogonal
(Weisman et al., 2017a), but also when children reason about a wider
range of artifacts and animals and factors are allowed to correlate. In
both cases, older children's mental capacity attributions revealed an
adult-like distinction between physiological, social-emotional, and
perceptual-cognitive abilities---resonating with traditional notions of
BODY, HEART, and MIND, respectively (Weisman et al., 2017b).

Beyond replicating previous findings, the use of an oblique rotation
method also allowed us to examine the correlations among factor loadings
for each factor---one way to probe the similarities and conceptual
connections across these latent constructs. The BODY and HEART factors
were somewhat more strongly correlated (\(\phi\)=0.44, bootstrapped 95\%
CI: {[}0.29, 0.59{]}) than were BODY and MIND (\(\phi\)=0.33 {[}0.14,
0.52{]}) or HEART and MIND (\(\phi\)=0.26 {[}0.08, 0.44{]}). This hints
at a possible conceptual connection that we were previously unaware of:
Although physiological and social-emotional abilities seem to emerge
from distinct latent constructs in older children's reasoning, there may
be a privileged relationship between BODY and HEART.

\subsubsection{EFA: Younger children
(4-6y)}\label{efa-younger-children-4-6y}

EFA revealed 2 factors that met our retention criteria. (Parallel
analysis also yielded 2 factors; BIC was minimized by a 1-factor
solution.) After rotation, the first factor included both physiological
sensations and emotions: It was the dominant factor for such items as
\emph{get hungry, feel sick--like when you feel like you might throw up,
feel happy, feel sad}. The second factor corresponded to
perceptual-cognitive abilities: It was the dominant factor for such
items as \emph{sense temperatures, remember things, sense whether
something is close by or far away, feel guilty}. These factors were
rather highly correlated (\(\phi\)=0.62 {[}0.34, 0.65{]}). See Figure 1
(columns 4-5).

\begin{CodeChunk}
\begin{figure*}[tb]

{\centering \includegraphics{figs/figure1-1} 

}

\caption[Factor loadings from exploratory factor analyses]{Factor loadings from exploratory factor analyses. Items are grouped according to their dominant factor (the factor with the strongest factor loading) among older children (7-9y). The percent of variance explained by each factor (after factor retention and oblimin rotation) is listed at the bottom of each column.}\label{fig:figure1}
\end{figure*}
\end{CodeChunk}

These results suggest both similarities and differences relative to the
conceptual structure that older children (7-9y) appear to share with
adults. First, the similarities. Like older children, younger children's
responses were characterized by tight correlations among a suite of
perceptual and cognitive capacities that we have labeled MIND. Indeed,
younger children's MIND factor was highly congruent with older
children's MIND factor (Tucker's \emph{\(r_c\)}=0.79), highlighting one
aspect of the latent structure underlying younger children's responses
that resonates with adult intuitions.

In another sense, however, this conceptual structure is quite different
from older children and adults: In contrast to the distinction between
physiological abilities (BODY) and social-emotional abilities (what
we've called HEART) that seems to characterize older children's and
adults' mental capacity attributions, younger children's responses
suggest an \emph{integrated} category encompassing both physiological
and social-emotional abilities (BODY-HEART).

One indication of this conceptual integration or blurring is that this
factor was moderately congruent both with older children's BODY factor
(\emph{\(r_c\)}=0.75) and with older children's HEART factor
(\emph{\(r_c\)}=0.68), suggesting a resonance with both of these more
adult-like latent constructs. Indeed, among the mental capacities that
load strongly on this factor (loadings \(\geq\) 0.60) were both
physiological sensations (\emph{get hungry, feel sick, smell things})
and social-emotional experiences (\emph{feel happy, feel sad, feel
proud, get angry, feel love, get hurt feelings}, ).

Is there a nascent distinction between BODY and HEART that has somehow
been obscured in this analysis? If so, we might expect that forcibly
extracting three factors from younger children's responses would yield a
similar structure to that observed among older children. To explore this
possibility, we conducted an additional EFA, manually specifying that
the solution should have three factors.

This analysis yielded only middling evidence for a nascent division
between physiological vs.~social-emotional abilities among younger
children; see Figure 1 (columns 6-8). Forced Factor 1 was congruent with
older children's BODY factor (Tucker's \(r_c\)=0.86), while forced
Factor 2 was congruent with older children's HEART factor (Tucker's
\(r_c\)=0.82). But it is not clear that ``physiological
vs.~social-emotional'' is the best way to characterize the differences
between these first two (forced) factors. In particular, the fact that
the strongest-loading items for the first factor are negatively valenced
(\emph{get angry, get hungry, get hurt feelings}) while the
strongest-loading items for the second factor are positively valenced
(\emph{feel happy, feel love, feel proud}) suggests that emotional
valence might be a more salient distinction in this age group. This is
in line with recent work suggesting that valence is a particularly
important feature of emotion concept representations for young children
(Nook, Sasse, Lambert, McLaughlin, \& Somerville, 2017).

\subsubsection{Exploratory analysis: Differentiation at the participant
level}\label{exploratory-analysis-differentiation-at-the-participant-level}

How might age-related differences in conceptual structure manifest in
individual children's mental capacity attributions? We now present an
exploratory analysis of the differentiation of BODY, HEART, and MIND
categories by individual children---a kind of non-parametric,
participant-level analysis meant to parallel the EFAs reported above.

We based this analysis on the intuition that a child who differentiates
clearly between two categories (e.g., BODY vs.~HEART) will evaluate
mental capacities related to these categories somewhat independently.
Such a child will sometimes end up endorsing mental capacities in one
category while rejecting mental capacities in the other (e.g., endorsing
most BODY items but rejecting most HEART items for whichever character
they are evaluating). By contrast, a child who does not differentiate
between these categories might be more likely to endorse or reject
across the board (e.g., endorsing equal numbers of BODY and HEART items
for their character). Of course, depending on the target character they
happen to be evaluating, even children with clearly differentiated
categories might end up endorsing equal numbers of capacities in both.
But if the differentiation of two categories becomes stronger over
development, we might expect to see that, on average, the difference in
the number of endorsements between these categories is greater for older
than younger children.

With these intuitions in mind, we used older children's EFA results to
choose sets of mental capacities to represent the categories BODY,
HEART, and MIND; see Figure 1. For each category, we included only items
that (1) loaded more strongly on that factor than on others and (2) were
among the 6 strongest-loading items for that
factor.\footnote{Two items, \textit{feel happy} and \textit{get angry}, were dropped from this analysis, because they were not in the top 6 items for any factor.}
We tallied the number of ``endorsements'' (responses of \emph{yes} or
\emph{kinda}) for the items included in each factor; each child could
endorse 0-6 capacities for each category.

We then examined differences in the number of endorsements between each
pair of categories: HEART minus BODY, MIND minus BODY, and MIND minus
HEART. Distributions of differences in endorsements across pairs of
categories for each age group are presented in Figure 2, with
descriptive statistics in Table 1 and comparisons of variances and
central tendencies in Table 2.

\begin{CodeChunk}
\begin{figure}[tb]
\includegraphics{figs/figure2-1} \caption[Distributions of differences between categories in the number of mental capacities endorsed by individual children (by age group)]{Distributions of differences between categories in the number of mental capacities endorsed by individual children (by age group).}\label{fig:figure2}
\end{figure}
\end{CodeChunk}

First we consider the differentiation of HEART vs.~BODY. Echoing the
contrast in factor structure revealed by EFA---in which HEART and BODY
were distinct factors among older children but integrated into a single
factor among younger children---older children seem to have been more
likely to differentiate strongly between these categories in their
mental capacity endorsements, as illustrated by the greater number of
older children with difference scores \(\gg\) 0 or \(\ll\) 0 (Fig. 2,
top). In line with this, the variance of younger children's difference
scores was lower than the variance of older children's difference scores
(see Tables 1-2).

Next, we consider MIND vs.~BODY. Recall that MIND was identified as a
latent construct distinct from BODY and HEART in EFAs for both age
groups. Echoing this, children in both age groups differentiated between
these categories, as illustrated by the many participants with
difference scores \(\gg\) 0 or \(\ll\) 0 (Fig. 2, middle). However,
these distributions of difference scores differed in their central
tendency (Tables 1-2): Younger children tended to attribute more BODY
than MIND capacities, while older children tended to attribute more MIND
than BODY capacities.

Finally, we consider the differentiation of HEART vs.~MIND. Again, many
children in both age groups differentiated between these categories
(Fig. 2, bottom). In this case, however, older children were especially
likely to have extreme difference scores, as reflected by the difference
in variance between age groups (Tables 1-2). These distributions also
differed in their central tendency: Younger children tended to attribute
more HEART than MIND capacities, while older children tended to
attribute more MIND than HEART capacities.

\begin{table}[ht]
\centering
\begin{tabular}{l|rrrrrr}
  \hline
 & n & med. & mean & var. & skew. & kurt. \\ 
  \hline
4-6y: B-H & 124 & 0.00 & -0.37 & 2.37 & -0.49 & 0.46 \\ 
  7-9y: B-H & 123 & 0.00 & -0.54 & 3.50 & 0.11 & -0.26 \\ 
  4-6y: B-M & 124 & 0.00 & -0.53 & 3.70 &  0.45 & 0.47 \\ 
  7-9y: B-M & 123 & 0.00 &  0.37 & 4.51 & 0.48 & -0.06 \\ 
  4-6y: M-H & 124 & 0.00 & -0.16 & 3.10 &  0.09 & 0.31 \\ 
  7-9y: M-H & 123 & 0.00 &  0.90 & 5.32 & 0.18 &  0.15 \\ 
   \hline
\end{tabular}
\caption{Descriptive statistics for the distributions pictured in Figure 1.} 
\end{table}

\begin{table}[ht]
\centering
\begin{tabular}{l|rr|rr|rr}
  \hline
 & W & p & t & p & K\verb|^|2 & p \\ 
  \hline
B-H & 8212.50 & 0.28 &  0.76 & 0.45 & 4.63 & 0.03 \\ 
  M-B & 5805.50 & 0.00 & -3.48 & 0.00 & 1.21 & 0.27 \\ 
  M-H & 5449.50 & 0.00 & -4.07 & 0.00 & 8.83 & 0.00 \\ 
   \hline
\end{tabular}
\caption{Testing differences between age groups: Results of Mann-Whitney-Wilcoxon tests ($  {W}$) and Welch's tests ($  {t}$), for comparing central tendencies, and Bartlett's tests ($    {K^2}$) for comparing variances.} 
\end{table}

\section{General discussion}\label{general-discussion}

We set out to investigate the development of reasoning about mental
life, with the goal of comparing the conceptual structures that underlie
mental capacity attributions in early childhood (4-6y) vs.~middle
childhood (7-9y). To this end, we examined patterns in children's
attributions of a wide range of mental capacities to various target
characters.

Two key findings emerged from this study. Both younger and older
children treated perceptual-cognitive abilities (MIND) as a distinct
component of mental life: Abilities to detect, store, and use
information about the environment travelled together in their
attributions, and were endorsed somewhat independently from
physiological or social-emotional abilities. But while older children
differentiated between physiological and social-emotional abilities as
two distinct components of mental life (BODY vs.~HEART), younger
children were more likely to group them together into a single
integrated category (BODY-HEART). These two findings---the similarity in
younger vs.~older children's understanding of MIND and the difference in
their understanding of BODY vs.~HEART---emerged both in our planned
comparison of the correlation structures underlying responses at the
group level (via factor analyses) and in our exploratory analysis of
differences in endorsements between categories at the individual level.

Beyond this, this exploratory analysis surfaced additional age-related
differences that were not evident from the factor analyses alone.

First, it revealed different biases in mental capacity attributions
across the two age groups: While older children frequently endorsed MIND
capacities in the absence of BODY or HEART capacities, younger children,
if anything, showed the opposite bias, particularly in their endorsement
of BODY capacities in the absence of MIND. This hints at the possibility
that children of different ages might perceive different kinds of
relationships among BODY, HEART, and MIND. In particular, younger
children's tendency to endorse BODY in the absence of MIND is consistent
with the idea that the physiological abilities characteristic of
biological animals are a necessary precondition for perceptual-cognitive
abilities---but older children's endorsement patterns suggest that they
might consider it possible for an entity to have social-emotional or
perceptual-cognitive abilities in the absence of biological animacy.
This an issue of particular importance in the modern world, in which
children are increasingly encountering ``smart,'' ``social''
technologies intended to convey cognitive prowess and social-emotional
presence. Our results suggest that children of different ages might have
different intuitions about the mental lives that such technological
beings might be capable of, with older children (7-9y) being
particularly open to the possibility of non-biological HEARTs and MINDs.

Second, older children appear to have differentiated more strongly than
younger children not only between HEART and BODY (as revealed by factor
analysis), but also between HEART and MIND, suggesting that one of the
important questions that children appear to be grappling with during
this period in development is how to make sense of emotional experience
in relation to the body and the mind. While the category of ``emotions''
might seem natural to many readers, there is much debate among affective
scientists and cultural psychologists about whether emotions are in fact
a universal natural kind (e.g., Barrett, 2006; Russell, 1991;
Wierzbicka, 1994). From the perspective of the BODY-HEART-MIND
framework, emotional experiences frequently center on physiological
sensations, such as an aching heart, a pit in the stomach, shaking
hands, or flushed cheeks---but emotional life is also fundamentally
cognitive, involving the perception, appraisal, and reframing of
experiences in ways that reshape the experience itself (e.g., Gross,
2015). How does any ordinary person come to distinguish ``emotions''
from other physiological and cognitive processes? What kinds of personal
experiences, social pressures, language demands, and cultural forces
encourage US children to abstract away a third category of mental life,
which seems to draw on both physiological and cognitive abilities while
somehow constituting a third kind of ``thing''? And what other
categories of mental life might a person come to see if they grew up in
a different context?

The developmental and cultural origins of ordinary people's
understanding of mental life are fascinating questions, deserving of
further research. We hope that the current study can lay the foundation
for these investigations.

\section{Acknowledgements}\label{acknowledgements}

This material is based upon work supported by the NSF GRFP under Grant
No. DGE-114747, and by a William R. \& Sara Hart Kimball Stanford
Graduate Fellowship. Thanks to C. Field, L.D. Brenner, O. Homer, C. Xie,
J.N. Sullivan, L. Saylor, B. Radecki, A. Yu, \& A. Kelly; the Tech
Museum of Innovation, the Palo Alto Junior Museum \& Zoo, and Bing
Nursery School; and the families who participated in this research.

\section{References}\label{references}

Barrett, L.F. (2006). Are emotions natural kinds? \emph{Perspect Psychol
Sci}, \emph{1}(1), 28--58.

Carey, S. (1985). \emph{Conceptual Change in Childhood}. Cambridge, MA:
MIT Press.

Flavell, J.H. (1999). Cognitive development: Children's knowledge about
the mind. \emph{Annu Rev Psychol}, \emph{50}, 21--45.

Gelman, S., \& Opfer, J. (2002). Development of the animate-inanimate
distinction. In U. Goswami (Ed.), \emph{Blackwell Handbook of Childhood
Cognitive Development} (pp.~151--166). Blackwell Publishers Ltd.

Gray, H.M., Gray, K., \& Wegner, D.M. (2007). Dimensions of mind
perception. \emph{Science}, \emph{315}(5812), 619.

Gross, J.J. (2015). Emotion regulation: Current status and future
prospects. \emph{Psychol Inq}, \emph{26}(1), 1--26.

Medin, D., Waxman, S., Woodring, J., \& Washinawatok, K. (2010).
Human-centeredness is not a universal feature of young children's
reasoning: Culture and experience matter when reasoning about biological
entities. \emph{Cog Dev}, \emph{25}(3), 197--207.

Nook, E.C., Sasse, S.F., Lambert, H.K., McLaughlin, K.A., \& Somerville,
L.H. (2017). Increasing verbal knowledge mediates development of
multidimensional emotion representations. \emph{Nat Hum Behav},
\emph{1}(12), 881--89.

Piaget, J. (1929). \emph{The Child's Conception of the World}. London:
Routledge \& Kegan Paul Ltd.

Russell, J.A. (1991). Culture and the categorization of emotions.
\emph{Psychol Bull}, \emph{110}(3), 426--50.

Weisman, K., Dweck, C.S., \& Markman, E.M. (2017a). Children's
intuitions about the structure of mental life. In \emph{Proc Annu
Meeting Cog Sci Soc} (pp.~1333--38).

Weisman, K., Dweck, C.S., \& Markman, E.M. (2017b). Rethinking people's
conceptions of mental life. \emph{Proc Natl Acad Sci}, \emph{114}(43),
11374--79.

Wellman, H.M. (2015). \emph{Making Minds: How Theory of Mind Develops}.
New York, NY: Oxford University Press.

Wierzbicka, A. (1994). Emotion, language, and cultural scripts. In
\emph{Emotion and culture: Empirical studies of mutual influence}.
(pp.~133--96).

\setlength{\parindent}{-0.1in} \setlength{\leftskip}{0.125in} \noindent

\end{document}
