% Template for Cogsci submission with R Markdown

% Stuff changed from original Markdown PLOS Template
\documentclass[10pt, letterpaper]{article}

\usepackage{cogsci}
\usepackage{pslatex}
\usepackage{float}
\usepackage{caption}

% amsmath package, useful for mathematical formulas
\usepackage{amsmath}

% amssymb package, useful for mathematical symbols
\usepackage{amssymb}

% hyperref package, useful for hyperlinks
\usepackage{hyperref}

% graphicx package, useful for including eps and pdf graphics
% include graphics with the command \includegraphics
\usepackage{graphicx}

% Sweave(-like)
\usepackage{fancyvrb}
\DefineVerbatimEnvironment{Sinput}{Verbatim}{fontshape=sl}
\DefineVerbatimEnvironment{Soutput}{Verbatim}{}
\DefineVerbatimEnvironment{Scode}{Verbatim}{fontshape=sl}
\newenvironment{Schunk}{}{}
\DefineVerbatimEnvironment{Code}{Verbatim}{}
\DefineVerbatimEnvironment{CodeInput}{Verbatim}{fontshape=sl}
\DefineVerbatimEnvironment{CodeOutput}{Verbatim}{}
\newenvironment{CodeChunk}{}{}

% cite package, to clean up citations in the main text. Do not remove.
\usepackage{cite}

\usepackage{color}

% Use doublespacing - comment out for single spacing
%\usepackage{setspace}
%\doublespacing


% % Text layout
% \topmargin 0.0cm
% \oddsidemargin 0.5cm
% \evensidemargin 0.5cm
% \textwidth 16cm
% \textheight 21cm

\title{Folk philosophy of mind: Changes in conceptual structure between 4-9y of
age}


\author{{\large \bf Kara Weisman} \\ \texttt{kweisman@stanford.edu} \\ Department of Psychology \\ Stanford University \And {\large \bf Carol S. Dweck} \\ \texttt{dweck@stanford.edu} \\ Department of Psychology \\ Stanford University \And {\large \bf Ellen M. Markman} \\ \texttt{markman@stanford.edu} \\ Department of Psychology \\ Stanford University}

\begin{document}

\maketitle

\begin{abstract}
We explored children's developing understanding of mental life using a
novel approach to track changes in conceptual structure from the bottom
up by analyzing patterns of mental capacity attributions. US children
(\emph{n}=247) evaluated elephants, goats, mice, birds, beetles, teddy
bears, dolls, robots, and computers on a range of mental capacities,
allowing us to assess which attributions ``go together'' and how these
conceptual connections might develop over early and middle childhood.
Replicating previous studies with adults and older children, an
exploratory factor analysis of older children's (7-9y) responses
revealed a three-way distinction between physiological abilities (e.g.,
hunger, smell), social-emotional abilities (e.g., guilt, embarrassment),
and perceptual-cognitive abilities (e.g., choice, memory), corresponding
to traditional notions of BODY, HEART, and MIND. Hints of this three-way
distinction emerged among younger children (4-6y), but younger children
appeared to perceive markedly stronger connections among physiological
and social-emotional abilities, while clearly distinguishing both from
the MIND.

\textbf{Keywords:}
mind perception; conceptual change; lay biology; lay psychology;
cognitive development.
\end{abstract}

\section{Introduction}\label{introduction}

From early in life, attributions of mental capacities govern our
interactions with other beings and inform our judgments about their
moral status. In order to understand, predict, and coordinate with
others, we make inferences about their thoughts, feelings, and other
aspects of mental life.

Developmental and cognitive psychologists have made great progress in
understanding how people make sense of other minds by postulating
distinct representations of such categories as ``perceptions,''
``beliefs,'' ``desires,'' and ``emotions.'' These categories have been
incredibly useful---but do they correspond to children's own developing
understanding of the structure of mental life?

After all, mental life is extremely complex. Consider a few examples of
the many dimensions that might organize this conceptual space: Some
mental capacities are closely related to specific bodily organs (e.g.,
vision, hunger), and others less obviously so (e.g., belief); some are
positively or negatively valenced (e.g., pain, happiness), and others
more neutral or variable (e.g., smell, thought); some involve taking in
information about the environment, while others involve storing,
updating, or using that information to bring about changes in the
external world; some are broadly similar across species, and others may
be unique to humans. How do people of different ages conceive of the
connections and distinctions among mental states, and how does this
conceptual structure shape their understanding of the various humans,
animals, and technologies in their world?

In recent studies, we have set out to derive this conceptual structure
empirically, using an approach---exploratory factor analysis---that
reveals the conceptual connections and distinctions underlying
participants' responses from the bottom up. Inspired by Gray et al.'s
(2007) work on the ``dimensions of mind perception,'' we first used this
bottom-up approach to analyze patterns of mental capacity attributions
among US adults. Across several studies, three suites of mental
capacities consistently emerged: (1) A suite of capacities related to
the BODY, including physiological sensations and self-initiated
behaviors; (2) a suite of capacities related to the HEART, including
social-emotional experiences and moral agency; and (3) a suite of
capacities related to the MIND, including perceptual-cognitive abilities
and goal pursuit (Weisman et al., 2017b). A further study with 7- to
9-year-old US children using an adapted version of this experimental
paradigm suggested that this distinction between BODY, HEART, and MIND
might be in place by middle childhood (Weisman et al., 2017a).

Here we extend this approach down to preschool-age children to explore
the earlier development of this conceptual system. The preschool years
are considered to be a time of rapid conceptual change in the domain of
lay psychology, as evidenced by dramatic shifts in children's abilities
to take others' perspectives, represent false beliefs, and integrate
representations of intentions and outcomes in evaluating moral
responsibility (for reviews, see Flavell, 1999; Wellman, 2015). The
period between 4-10y of age has also been the focus of a rich tradition
of work on lay biology extending back nearly a century (e.g., Carey,
1985; Medin et al., 2010; Piaget, 1929). All of these accounts make the
case that becoming a sophisticated reasoner---and particularly a
sophisticated social reasoner---requires substantial refinement of one's
representations of others' experiences, beliefs, desires, and needs.
Might these refinements include shifts in children's intuitions about
the fundamental components of mental life?

In the current paper, we examine snapshots of this conceptual structure
at two points in development (ages 4-6y and 7-9y) within a well-studied
cultural context (the US). We aim to assess the similarities and
differences in younger vs.~older children's representations of
sensations, perceptions, beliefs, thoughts, desires, emotions, and other
aspects of mental life. This is the first step in developing a more
nuanced account of how this core aspect of folk philosophy of mind might
emerge and change over the course of early and middle childhood.

\section{Study}\label{study}

We based our experimental paradigm on our previous work with children
ages 7-9y (Weisman et al., 2017a), in which children evaluated a target
character on a variety of mental capacities using a 3-point response
scale (\emph{no}, \emph{kinda}, \emph{yes}). Although a 3-point scale is
not optimal for factor analyses, it enabled children as young as 4y to
answer questions comfortably and complete many trials. Including a wide
age range while maintaining a within-subjects design was our top
priority for the planned factor analyses.

As in previous work with adults (Weisman et al., 2017b, Study 4), we
asked each participant to judge the mental capacities of one target
character out of a set of familiar entities. We included both animals
and artifacts in this set, with an eye toward exploring age-related
differences in the relationship between attributions of biological
animacy and mental life (see, e.g., Carey, 1985; Gelman \& Opfer, 2002).
For animals, we included both mammals of different sizes and
relationships to humans (elephant, goat, mouse) and non-mammals (bird,
beetle), to represent a range of creatures that might vary in their
perceived mental capacities. For artifacts, we included both
anthropomorphic toys (teddy bear, doll) and ``smart'' technologies
(robot, computer), which present different kinds of challenges in terms
of grappling with the relationship between animacy, pretense, and mental
life. Most critically for our bottom-up approach, we expected these
target characters to be perceived as having very different mental lives:
Robots, for example, are generally thought to have a different set of
mental capacities than, say, goats (Weisman et al., 2017b). This allowed
us to address the following question: When different characters are
thought to have different profiles of mental capacities, which
capacities ``go together''?

\subsection{Methods}\label{methods}

\subsubsection{Participants}\label{participants}

247 children participated in this study, which took place in the San
Francisco Bay Area. Our planned sample size was 120 older and 120
younger children, but we also retained a handful of extra participants
who completed the study on the final day of data collection for each
group. Older children (\emph{n}=123) ranged in age from 7.09-9.99y
(median: 8.57y), and participated at local museums; the median study
duration for this group was 2.70min. Younger children (\emph{n}=124)
ranged in age from 4.00-6.99y (median: 5.03y), and participated either
at their preschool (68\%) or at a museum (32\%); the median study
duration for this group was 3.58min. An additional 7 children
participated but were excluded for being outside the target age range.

We grouped children into two age groups because our primary planned
analysis---exploratory factor analysis---is a group-level analysis of
the consensual conceptual structure, and is not designed to model
continuous participant-level variables like exact age. Our goal in this
study was to examine discrete ``snapshots'' of this conceptual structure
at two points in this continuous developmental trajectory.

\subsubsection{Materials and procedure}\label{materials-and-procedure}

Participants were assigned to evaluate one of the following characters:
elephant, goat, mouse, bird, beetle, teddy bear, doll, robot, computer
(\emph{n}=10-16 per character, per age group). Participants were
assigned to condition randomly, with two exceptions: The doll and teddy
bear conditions were run last for older children (but included in the
initial randomization scheme for younger children); and toward the end
of data collection for each age group children were assigned to
conditions that had the fewest participants. A vivid, high-resolution
photo of the target character in a naturalistic context (e.g., a
humanoid robot in an office) and a label (e.g., a robot) were displayed
on a computer screen for the duration of the study.

Instructions were identical to previous work with children (Weisman et
al., 2017a), focusing on the idea that we wanted to know what children
thought (e.g.) ``{[}robots{]} can do and can not do.'' Children rated
the target character on 20 mental capacities, presented in a random
order for each participant. On each trial, children responded \emph{no},
\emph{kinda}, or \emph{yes} to the question ``Do you think a {[}robot{]}
can\ldots{}?'' The experimenter read the instructions and the first
question out loud. Older children were then given the option of reading
and responding to subsequent questions on their own using the
experimenter's laptop, which some but not all participants opted to do.
All younger children heard all questions read aloud by the experimenter
and responded verbally.

The 20 mental capacities were a subset of the 40 items used in previous
work with children (Weisman et al., 2017a), including physiological
sensations, emotional experiences, perceptual abilities, cognitive
skills, capacities related to autonomy or agency, and social abilities;
see Figure 1. As in previous work, each mental capacity was associated
with a short, preset definition. Children were encouraged at the
beginning of the study to ask questions if they did not know what a word
meant, in which case they were given these definitions.

\subsubsection{Data preparation}\label{data-preparation}

We scored responses of \emph{no} as 0, \emph{kinda} as 0.5, and
\emph{yes} as 1. We planned to drop trials with response times that were
faster than a preset criterion of 250ms, but there were none. We
retained participants regardless of skipped trials (0 trials among older
children, 30 trials among younger children). Overall, none of older
children's trials and only 1.21\% of younger children's trials were
missing data.

\subsubsection{Planned analyses}\label{planned-analyses}

Following previous work, we conducted exploratory factor analyses (EFA)
to reveal the latent structure underlying participants' mental capacity
attributions, collapsing across characters and using Pearson
correlations to find minimum residual solutions. We first examined
maximal (14-factor) solutions to determine how many factors to extract,
using the following preset retention criteria (identical to Weisman et
al., 2017a, 2017b): Each factor must have an eigenvalue
\textgreater{}1.0 and individually account for \textgreater{}5\% of the
shared variance before rotation; and each must be the ``dominant''
factor (have the strongest absolute factor loading) for \(\geq\) 1
mental capacity after rotation. We used an oblique rotation (oblimin)
here because it allows us to examine correlations among factors; note,
however, that constraining factors to be orthogonal (via varimax
rotation) yielded very similar latent structures. We compared this
factor retention approach to two common alternatives: parallel analysis,
which compares the observed correlation structure to the correlation
structure arising from random datasets of the same size; and minimizing
the Bayesian Information Criterion (BIC), which is one method of
optimizing both goodness of fit and parsimony.

\subsection{Results and discussion}\label{results-and-discussion}

We first assess the conceptual replication of our previous work with 7-
to 9-year-old children by conducting EFA of older children's responses.
We then examine this conceptual system at an earlier point in
development via EFA of younger children's responses. Finally, we present
a post-hoc analysis of individual children's endorsements of three
categories of mental capacities: physiological, social-emotional, and
perceptual-cognitive. This provides a more intuitive picture of the EFA
results and sheds new light on how the co-occurrence of endorsements
across these three categories might vary with age.

\subsubsection{EFA: Older children
(7-9y)}\label{efa-older-children-7-9y}

EFA revealed 3 factors that met our retention criteria. Alternative
approaches to factor retention---parallel analysis and minimizing
BIC---also yielded 3 factors. See Figure 1 (columns 1-3) for the full
results of this analysis.

After rotation, the first factor corresponded primarily to physiological
sensations and other experiences related to biological needs and
physical survival. It was the dominant factor for such items as
\emph{get hungry, smell things, feel scared, feel pain}. The second
factor corresponded primarily to social-emotional experiences. It was
the dominant factor for such items as \emph{feel guilty, feel
embarrassed, feel proud, get hurt feelings}. The third factor
corresponded primarily to perceptual and cognitive abilities to detect,
store, and make use of information about the environment. It was the
dominant factor for such items as \emph{figure out how to do things,
make choices, remember things, sense temperatures}.

This provides a conceptual replication of our previous work with this
age group, suggesting that this conceptual structure emerges not only
when children are asked to reason about controversial ``edge cases''
(beetles and robots) and factors are constrained to be orthogonal
(Weisman et al., 2017a), but also when children reason about a wider
range of artifacts and animals and factors are allowed to correlate. In
both cases, older children's mental capacity attributions revealed an
adult-like distinction between physiological, social-emotional, and
perceptual-cognitive abilities---resonating with traditional notions of
BODY, HEART, and MIND, respectively (Weisman et al., 2017b).

Beyond replicating previous findings, the use of an oblique rotation
method also allowed us to examine the correlations among factor loadings
for each factor---one way to probe the similarities and conceptual
connections across these latent constructs. The BODY and HEART factors
were somewhat more strongly correlated (\(\phi\)=0.48, bootstrapped 95\%
CI: {[}0.28, 0.67{]}) than were BODY and MIND (\(\phi\)=0.28 {[}0.10,
0.47{]}) or HEART and MIND (\(\phi\)=0.23 {[}0.09, 0.37{]}). This hints
at a possible conceptual connection that we were previously unaware of:
Although physiological and social-emotional abilities seemed to emerge
from distinct latent constructs in older children's reasoning, there may
have been a privileged relationship between BODY and HEART.

\subsubsection{EFA: Younger children
(4-6y)}\label{efa-younger-children-4-6y}

Again, 3 factors met our preset retention criteria (Fig. 1, col. 4-6).
After rotation, the first factor included both physiological sensations
(BODY) and emotions (HEART): It was the dominant factor for such items
as \emph{get angry, get hungry, get hurt feelings, smell things}. The
second factor primarily included emotions (HEART): It was the dominant
factor for such items as \emph{feel happy, feel love, feel proud, feel
scared}. The third factor corresponded to perceptual-cognitive abilities
(MIND): It was the dominant factor for such items as \emph{sense
temperatures, remember things, sense whether something is close\ldots{},
feel guilty}. Again, the first and second factors were somewhat more
strongly correlated (\(\phi\)=0.45 {[}0.35, 0.56{]}) than were first and
third (\(\phi\)=0.34 {[}0.13, 0.55{]}) or the second and third
(\(\phi\)=0.28 {[}0.07, 0.48{]}).

Meanwhile, parallel analysis suggested a 2-factor solution (Fig. 1, col.
7-8). After rotation, the first factor included both physiological
sensations (BODY) and emotions (HEART; e.g., \emph{get hungry, feel
sick\ldots{}, feel happy, feel sad}), while the second factor
corresponded to perceptual-cognitive abilities (MIND; e.g., \emph{sense
temperatures, remember things, sense whether something is close\ldots{},
feel guilty}). The two factors were moderately correlated (\(\phi\)=0.52
{[}0.40, 0.64{]}).

BIC was minimized by a 1-factor solution (not reported).

\begin{CodeChunk}
\begin{figure*}[tb]

{\centering \includegraphics{figs/figure1-1} 

}

\caption[Factor loadings from exploratory factor analyses]{Factor loadings from exploratory factor analyses. Items are ordered according to their dominant factor (the factor with the strongest factor loading) among older children (7-9y). The percent of shared variance explained by each factor (after factor retention and oblimin rotation) is listed at the bottom of each column.}\label{fig:figure1}
\end{figure*}
\end{CodeChunk}

Taken together, these results suggest both similarities and differences
relative to the conceptual structure that older children (7-9y) appeared
to share with adults in previous work.

First, like older children, younger children's responses were
characterized by strong correlations among a suite of perceptual and
cognitive capacities that we have labeled MIND. Indeed, younger
children's perceptual-cognitive factor was highly congruent with older
children's MIND factor, both in the 3-factor solution (Tucker's
\emph{\(r_c\)}=0.81) and in the 2-factor solution (\emph{\(r_c\)}=0.79).
This highlights one aspect of the latent structure underlying younger
children's responses that resonates with the intuitions of older
children and adults.

But in contrast to the clear distinction between physiological abilities
and social-emotional abilities that characterized older children's
mental capacity attributions, younger children's responses suggest that
they perceived physiological and social-emotional abilities to be more
closely integrated and the line between them to be more blurred.

One indication of this blurring comes from the 2-factor solution
suggested by parallel analysis, in which a single BODY-HEART factor
emerged and was moderately congruent with both the BODY
(\emph{\(r_c\)}=0.75) and HEART (\emph{\(r_c\)}=0.68) factors of older
children. Among the mental capacities that loaded strongly (\(\geq\)
0.60) on this factor were both physiological sensations (\emph{get
hungry, feel sick\ldots{}, smell things}) and social-emotional
experiences (\emph{feel happy, feel sad, feel proud, get angry, feel
love, get hurt feelings}), suggesting that younger children perceived
physiological and social-emotional abilities to ``go together'' to a
considerable degree.

Even in the 3-factor solution suggested by our standard factor retention
protocol, the distinction between physiological and social-emotional
abilities was somewhat blurred. While the first factor was highly
congruent with older children's BODY factor (\emph{\(r_c\)}=0.86), it
was also the dominant factor for several social-emotional items
(\emph{get angry, get hurt feelings, feel sad}). And while the second
factor was highly congruent with older children's HEART factor
(\emph{\(r_c\)}=0.82), there were several social-emotional items that
failed to load strongly on it (loadings \(\leq\) 0.30: \emph{get angry,
get hurt feelings, feel sad, feel guilty}). Stepping back, it is not
clear that ``physiological vs.~social-emotional'' is the best way to
characterize the differences between these two factors. In fact, given
that the strongest-loading items for the first factor were negatively
valenced (\emph{get angry, get hungry, get hurt feelings}) while the
strongest-loading items for the second factor were positively valenced
(\emph{feel happy, feel love, feel proud}), it seems plausible that the
more salient distinction among this age group may have been positive
vs.~negative valence, rather than BODY vs.~HEART. This is in line with
recent work suggesting that valence is a particularly important feature
of emotion concept representations for young children (Nook et al.,
2017).

Finally, the very fact that different approaches to factor retention
yielded different results is further evidence that, although we observed
some evidence for a nascent distinction between BODY and HEART among
younger children, this distinction was not as robust as it appeared to
be among older children or among adults in previous work.

\subsubsection{Exploratory analysis: Differentiation at the participant
level}\label{exploratory-analysis-differentiation-at-the-participant-level}

How might age-related differences in conceptual structure manifest in
individual children's mental capacity attributions? We now present an
exploratory analysis of the differentiation of BODY, HEART, and MIND
categories by individual children---a kind of non-parametric,
participant-level analysis meant to parallel the EFAs reported above.

We based this analysis on the intuition that a child who differentiates
clearly between two categories (e.g., BODY vs.~HEART) will evaluate
mental capacities related to these categories somewhat independently.
Such a child will sometimes end up endorsing mental capacities in one
category while rejecting mental capacities in the other (e.g., endorsing
most BODY items but rejecting most HEART items)---whereas a child who
does not differentiate between these categories might be more likely to
endorse or reject across the board (e.g., endorsing equal numbers of
BODY and HEART items). Of course, depending on the target character they
happen to evaluate, even children with clearly differentiated categories
might end up endorsing equal numbers of capacities in both. But if the
differentiation of two categories becomes stronger over development, we
might expect that, on average, the difference in the number of
endorsements between these categories would be greater for older than
younger children.

With these intuitions in mind, we used older children's EFA results to
choose sets of mental capacities to represent the categories BODY,
HEART, and MIND. For each category, we included only items that (1)
loaded more strongly on that factor than on others and (2) were among
the 6 strongest-loading items for that
factor.\footnote{Two items, \textit{feel happy} and \textit{get angry}, were dropped from this analysis, because they were not in the top 6 items for any factor.}
We tallied the number of ``endorsements'' (responses of \emph{yes} or
\emph{kinda}) for the items in each category; each child could endorse
0-6 capacities for each category. We then examined differences in the
number of endorsements between each pair of categories: HEART minus
BODY, MIND minus BODY, and MIND minus HEART. Distributions of
differences in endorsements across pairs of categories for each age
group are presented in Figure 2, with comparisons of variances and
central tendencies in Table 1.

\begin{CodeChunk}
\begin{figure}[tb]
\includegraphics{figs/figure2-1} \caption[Distributions of between-category differences in how many capacities each child, by age group]{Distributions of between-category differences in how many capacities each child, by age group.}\label{fig:figure2}
\end{figure}
\end{CodeChunk}

First we consider the differentiation of HEART vs.~BODY. Echoing the
contrast in factor structure revealed by EFA---in which HEART and BODY
were distinct factors among older children but seemed more integrated
among younger children---older children seem to have been more likely to
differentiate strongly between these categories in their mental capacity
endorsements, as illustrated by the greater number of older children
with difference scores \(\gg\) 0 or \(\ll\) 0 (Fig. 2, top). In line
with this, the variance of younger children's difference scores was
lower than the variance of older children's difference scores (see Table
1).

Next, we consider MIND vs.~BODY. Recall that MIND was identified as a
latent construct distinct from BODY and HEART in EFAs for both age
groups. Echoing this, children in both age groups differentiated between
these categories, as illustrated by the many participants with
difference scores \(\gg\) 0 or \(\ll\) 0 (Fig. 2, middle). However,
these distributions of difference scores differed in their central
tendency (Table 1): Younger children tended to attribute more BODY than
MIND capacities, while older children tended to attribute more MIND than
BODY capacities.

Finally, we consider HEART vs.~MIND. Again, many children in both age
groups differentiated between these categories (Fig. 2, bottom). In this
case, however, older children were especially likely to have extreme
difference scores, as reflected by the difference in variance between
age groups (Table 1). These distributions also differed in their central
tendency: Younger children tended to attribute more HEART than MIND
capacities, while older children tended to attribute more MIND than
HEART capacities.

\begin{table}[ht]
\centering
\begin{tabular}{l|rr|rr|rr}
  \hline
 & W & p & t & p & K\verb|^|2 & p \\ 
  \hline
B-H & 8212.50 & 0.28 &  0.76 & 0.45 & 4.63 & 0.03 \\ 
  M-B & 5805.50 & 0.00 & -3.48 & 0.00 & 1.21 & 0.27 \\ 
  M-H & 5449.50 & 0.00 & -4.07 & 0.00 & 8.83 & 0.00 \\ 
   \hline
\end{tabular}
\caption{Mann-Whitney-Wilcoxon ($   {W}$) and Welch's ($    {t}$) tests for comparing central tendencies and Bartlett's tests ($    {K^2}$) for comparing variances in difference scores across age groups.} 
\end{table}

\section{General discussion}\label{general-discussion}

We set out to investigate the development of reasoning about mental
life, with the goal of comparing the conceptual structures that underlie
mental capacity attributions in early childhood (4-6y) vs.~middle
childhood (7-9y) in a well-studied cultural context (the US). To this
end, we examined patterns in children's attributions of a wide range of
mental capacities to various target characters.

Two key findings emerged from this study. Both younger and older
children treated perceptual-cognitive abilities (MIND) as a distinct
component of mental life: Abilities to detect, store, and use
information about the environment travelled together in their
attributions, and were endorsed somewhat independently from
physiological or social-emotional abilities. But while older children
differentiated between physiological and social-emotional abilities as
two additional, distinct components of mental life (BODY vs.~HEART),
among younger children this distinction was less clear and less robust.
These two findings---the similarity in younger vs.~older children's
understanding of MIND and the difference in their understanding of BODY
vs.~HEART---emerged both in our planned comparison of the correlation
structures underlying responses at the group level (via EFA) and in our
exploratory analysis of differences in endorsements between categories
at the individual level.

Beyond this, this exploratory analysis surfaced two age-related
differences that were not evident from EFA alone.

First, it revealed different biases in mental capacity attributions
across the two age groups: While older children frequently endorsed MIND
capacities in the absence of BODY or HEART capacities, younger children,
if anything, showed the opposite bias, particularly in their endorsement
of BODY capacities in the absence of MIND. This hints at the possibility
that children of different ages might perceive different kinds of
relationships among BODY, HEART, and MIND.

In particular, younger children's tendency to endorse BODY in the
absence of MIND is consistent with the idea that the physiological
abilities characteristic of biological animals are a necessary
precondition for perceptual-cognitive abilities---but older children's
endorsement patterns suggest that they might consider it possible for an
entity to have social-emotional or perceptual-cognitive abilities in the
absence of biological animacy. This is an issue of particular importance
in the modern world, in which children are increasingly encountering
``smart,'' ``social'' technologies intended to convey cognitive prowess
and social-emotional presence. Our results suggest that children of
different ages might have different intuitions about the mental lives of
technological beings, with older children (7-9y) being particularly open
to the possibility of non-biological HEARTs and MINDs.

Second, older children appear to have differentiated more strongly than
younger children not only between HEART and BODY (as revealed by EFA),
but also between HEART and MIND, suggesting that one of the important
questions that children appear to be grappling with during this period
in development is how to make sense of emotional experience in relation
to the body and the mind.

While the category of ``emotions'' might seem natural to many readers,
there is much debate among affective scientists and cultural
psychologists about whether emotions are in fact a universal natural
kind (e.g., Barrett, 2006; Russell, 1991; Wierzbicka, 1994). From the
perspective of the BODY-HEART-MIND framework, emotional experiences
frequently center on physiological sensations, such as an aching heart,
a pit in the stomach, or flushed cheeks---but emotional life is also
fundamentally cognitive, involving the perception, appraisal, and
reframing of experiences in ways that reshape the experience itself
(e.g., Gross, 2015). How does a person come to distinguish ``emotions''
from other physiological and cognitive processes? What kinds of personal
experiences, social pressures, language demands, and cultural forces
encourage US children to abstract away a third category of mental life,
which seems to draw on both physiological and cognitive abilities while
somehow constituting a third kind of ``thing''? And what other
categories of mental life might a person come to see if they grew up in
a different context?

The developmental and cultural origins of ordinary people's
understanding of mental life are fascinating questions, deserving of
further research. Two important next steps will be to move from the
``snapshot'' approach taken here to considering development more
continuously, and to investigate how this aspect of folk philosophy of
mind unfolds in contexts outside of the US, where both the developmental
trajectory and the adult endpoint might be subject to different cultural
forces from the ones at play here. The current study lays the foundation
for such investigations.

\section{Acknowledgements}\label{acknowledgements}

This material is based upon work supported by the NSF GRFP under Grant
No. DGE-114747, and by a William R. \& Sara Hart Kimball Stanford
Graduate Fellowship. Thanks to C. Field, L.D. Brenner, O. Homer, C. Xie,
J.N. Sullivan, L. Saylor, B. Radecki, A. Yu, A. Kelly; the Tech Museum
of Innovation, the Palo Alto Junior Museum \& Zoo, Bing Nursery School;
and the families who participated in this research.

\section{References}\label{references}

Barrett, L.F. (2006). Are emotions natural kinds? \emph{Perspect Psychol
Sci}, \emph{1}(1), 28--58.

Carey, S. (1985). \emph{Conceptual Change in Childhood}. Cambridge, MA:
MIT Press.

Flavell, J.H. (1999). Cognitive development: Children's knowledge about
the mind. \emph{Annu Rev Psychol}, \emph{50}, 21--45.

Gelman, S., \& Opfer, J. (2002). Development of the animate-inanimate
distinction. In U. Goswami (Ed.), \emph{Blackwell Handbook of Childhood
Cognitive Development} (pp.~151--166). Blackwell Publishers Ltd.

Gray, H.M., Gray, K., \& Wegner, D.M. (2007). Dimensions of mind
perception. \emph{Science}, \emph{315}(5812), 619.

Gross, J.J. (2015). Emotion regulation: Current status and future
prospects. \emph{Psychol Inq}, \emph{26}(1), 1--26.

Medin, D., Waxman, S., Woodring, J., \& Washinawatok, K. (2010).
Human-centeredness is not a universal feature of young children's
reasoning: Culture and experience matter when reasoning about biological
entities. \emph{Cog Dev}, \emph{25}(3), 197--207.

Nook, E.C., Sasse, S.F., Lambert, H.K., McLaughlin, K.A., \& Somerville,
L.H. (2017). Increasing verbal knowledge mediates development of
multidimensional emotion representations. \emph{Nat Hum Behav},
\emph{1}(12), 881--89.

Piaget, J. (1929). \emph{The Child's Conception of the World}. London:
Routledge \& Kegan Paul Ltd.

Russell, J.A. (1991). Culture and the categorization of emotions.
\emph{Psychol Bull}, \emph{110}(3), 426--50.

Weisman, K., Dweck, C.S., \& Markman, E.M. (2017a). Children's
intuitions about the structure of mental life. In \emph{Proc Annu
Meeting Cog Sci Soc} (pp.~1333--38).

Weisman, K., Dweck, C.S., \& Markman, E.M. (2017b). Rethinking people's
conceptions of mental life. \emph{Proc Natl Acad Sci}, \emph{114}(43),
11374--79.

Wellman, H.M. (2015). \emph{Making Minds: How Theory of Mind Develops}.
New York, NY: Oxford University Press.

Wierzbicka, A. (1994). Emotion, language, and cultural scripts. In
\emph{Emotion and culture: Empirical studies of mutual influence}.
(pp.~133--96).

\setlength{\parindent}{-0.1in} \setlength{\leftskip}{0.125in} \noindent

\end{document}
